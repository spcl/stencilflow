% document type (note: we use scrartcl instead of article in order to support \subtitle}
\documentclass{scrartcl} 

% package imports
\usepackage[utf8]{inputenc}
\usepackage{natbib}

% document info
\title{Efficient Implementation of a High-Performance In-production Weather Model on FPGA}
\subtitle{Task description of the Bachelor thesis, ETH Zurich, supervised by Johannes de Fine Licht, Professor Torsten Hoefler}
\author{Andreas Kuster}
\date{}

% document start 
\begin{document}
\maketitle

\section*{Introduction}
Accurate and reliable weather forecast is of vital importance for a broad field of industries, as well as the general public. Highly regular and statically analyzable stencil operators \cite{label6} on structured grids  are used to numerically solve the partial differential equations of such weather prediction models. This allows optimizations for data re-use while minimizing the high demand of memory bandwidth \cite{label2,label5} on the FPGA (field-programmable gate array) platform. Our collaboration with MeteoSwiss \cite{label37} enables us to apply our theoretical optimization findings to the numerical weather prediction and regional climate model COSMO \cite{label15, label38}. By cooperating closely with the University of Paderborn \cite{label42}, we gain access to a clustered heterogeneous supercomputer containing 32 interconnected Stratix 10 FPGAs  \cite{label26} whereby we intend to figure out if FPGAs \cite{label33} are the optimal choice for future high-performance weather prediction simulations.

\section*{Research questions to answer (Q) / Development tasks to perform (D)}
\begin{enumerate}
    \item Manual analysis and formalization of the problem. (Q)
    \item Defining a suitable input representation. (Q)
    \item Automatic analysis of the input including computation of important characteristics such as latency, buffer requirements, etc. (D)
    \item Feasibility estimate. (Q)
    \item Formulation of optimization goals and constraints. (Q)
    \item Automatic optimization of the input according to the goals/constraints using a suitable solver. (D)
    \item Formalization of a FPGA simulation model and implementation of the software simulation for testing, debugging and performance metric measures. (Q,D)
    \item Manual implementation of the stencil chains on (single/multiple) FPGAs including performance optimizations in HLS \cite{label18}. (Q,D)
    \item Automatic code generation. (D)
    \item $[$Optional$]$ Porting and testing of the complete dynamical core of COSMO to the FPGA platform. (D)
\end{enumerate}

%\section*{}
%\begin{tabbing}
%  \hspace*{0.5\textwidth}\=\kill
%  Place/Date \> Place/Date \\ \\ \\ \\
%  Student's signature \> Supervisor's signature
% \end{tabbing}
%
% \newpage

% generate bibliography (note: unsrt numbers the paper in the ordering of occurrence)
\bibliographystyle{unsrt}
\bibliography{references}

\end{document}